%%%%%%%%%%%%%%%%%%%%%%%%%%%%%%%%%%%%%%%%%%%%%%%%%%%%%%%%%%%%%%%%%%%%%%
% writeLaTeX Example: A quick guide to LaTeX
%
% Source: Dave Richeson (divisbyzero.com), Dickinson College
% 
% A one-size-fits-all LaTeX cheat sheet. Kept to two pages, so it 
% can be printed (double-sided) on one piece of paper
% 
% Feel free to distribute this example, but please keep the referral
% to divisbyzero.com
% 
%%%%%%%%%%%%%%%%%%%%%%%%%%%%%%%%%%%%%%%%%%%%%%%%%%%%%%%%%%%%%%%%%%%%%%
% How to use writeLaTeX: 
%
% You edit the source code here on the left, and the preview on the
% right shows you the result within a few seconds.
%
% Bookmark this page and share the URL with your co-authors. They can
% edit at the same time!
%
% You can upload figures, bibliographies, custom classes and
% styles using the files menu.
%
% If you're new to LaTeX, the wikibook is a great place to start:
% http://en.wikibooks.org/wiki/LaTeX
%
%%%%%%%%%%%%%%%%%%%%%%%%%%%%%%%%%%%%%%%%%%%%%%%%%%%%%%%%%%%%%%%%%%%%%%

\documentclass[10pt,landscape]{article}
\usepackage{amssymb,amsmath,amsthm,amsfonts}
\usepackage{multicol,multirow}
\usepackage{calc}
\usepackage{ifthen}
\usepackage[landscape]{geometry}
\usepackage[colorlinks=true,citecolor=blue,linkcolor=blue]{hyperref}
\usepackage{xcolor}
\usepackage{listings}
\lstset{basicstyle=\ttfamily,
  showstringspaces=false,
  commentstyle=\color{red},
  keywordstyle=\color{blue}
}

\ifthenelse{\lengthtest { \paperwidth = 11in}}
    { \geometry{top=.5in,left=.5in,right=.5in,bottom=.5in} }
 {\ifthenelse{ \lengthtest{ \paperwidth = 297mm}}
  {\geometry{top=1cm,left=1cm,right=1cm,bottom=1cm} }
  {\geometry{top=1cm,left=1cm,right=1cm,bottom=1cm} }
 }
\pagestyle{empty}
\makeatletter
\renewcommand{\section}{\@startsection{section}{1}{0mm}%
                                {-1ex plus -.5ex minus -.2ex}%
                                {0.5ex plus .2ex}%x
                                {\normalfont\large\bfseries}}
\renewcommand{\subsection}{\@startsection{subsection}{2}{0mm}%
                                {-1explus -.5ex minus -.2ex}%
                                {0.5ex plus .2ex}%
                                {\normalfont\normalsize\bfseries}}
\renewcommand{\subsubsection}{\@startsection{subsubsection}{3}{0mm}%
                                {-1ex plus -.5ex minus -.2ex}%
                                {1ex plus .2ex}%
                                {\normalfont\small\bfseries}}
\makeatother
\setcounter{secnumdepth}{0}
\setlength{\parindent}{0pt}
\setlength{\parskip}{0pt plus 0.5ex}
% -----------------------------------------------------------------------

\title{Quick Guide to \texttt{kworkflow}}

\begin{document}

\raggedright
\footnotesize

\begin{center}
     \Large{\textbf{A quick guide to \texttt{kworkflow}}} \\
\end{center}
\begin{multicols}{3}
\setlength{\premulticols}{1pt}
\setlength{\postmulticols}{1pt}
\setlength{\multicolsep}{1pt}
\setlength{\columnsep}{2pt}

\section{What is \texttt{kw}?}
This set of scripts have a simple mission: reduces the environment and setup overhead for developing for GNU/Linux. Kw is composed of different scripts unified in a single interface after the installation, kw commands become available in the command line interface.

\section{Install and Uninstall}

Package Dependencies: \\
\texttt{apt install libguestfs-tools qemu qemu-kvm python-docutils rsync}

For installing, you just need to run: \texttt{./setup -i}. Below you can see all the available options:

\begin{tabular}{ll}
\emph{Option} & \emph{Description} \\
\texttt{--help,-h} & Display this usage message \\
\texttt{--install,-i} & Install kw \\
\texttt{--uninstall,-u} & Uninstall kw \\
\texttt{--completely-remove} & Completely remove everything \\
\texttt{--html} & Build kw's documentation as HTML \\
\end{tabular}

\section{Find help}

If you need help, you have many options. Via the command line, you can use:

\texttt{kw help} \\
\texttt{kw man}

Online: \\
\url{https://siqueira.tech/doc/kw/} \\
\url{https://github.com/kworkflow/kworkflow}

\section{Check codestyle}
\texttt{codestyle,c}: Apply checkpatch on directory, file, or patch.

Example:
\begin{lstlisting}[language=bash]
cd drivers/gpu/drm/amd/display/
kw c amdgpu_dm/amdgpu_dm_irq.h
\end{lstlisting}
 
\section{Find maintainers}
\begin{description}
\item [\texttt{maintainers,m}]: This command shows the maintainers of a given Kernel module.
\end{description}

Options:

\begin{description}
    \item [\texttt{--authors,-a}]: Return the maintainers and the mailing list. "-a" also prints files authors
\end{description}

Example:
\begin{lstlisting}[language=bash]
kw m drivers/gpu/drm/vkms/
kw m -a drivers/gpu/drm/vkms/
\end{lstlisting}

\section{Find string match}

\begin{description}
\item [\texttt{explore,e}]: The explore command is based on git grep. It can search for string matches in either the git repository contents or in the git log messages.
\end{description}

Options:

\begin{description}
    \item [\texttt{STRING [PATH]}]:
        Search for STRING based in PATH (./ by default) 
    \item [\texttt{"STR SRT" [PATH]}]:
        Search for strings
    \item [\texttt{--log STRING}]:
        Search for STRING on git log
\end{description}

Example:
\begin{lstlisting}[language=bash]
kw e "Atomic check stopped"
kw e "Atomic check stopped" drivers/gpu/drm/
kw e dm_crtc_get_scanoutpos drivers/gpu/drm/
kw e --log "-EINVAL if something gets wrong"
\end{lstlisting}

\section{Manage \texttt{.config} file}

\begin{description}
    \item [\texttt{configm,g}]: The ‘configm’ command manages different versions of the project’s ‘.config’ file. It provides the save, load, remove, and list operations of such files.
\end{description}

Options:

\begin{description}
\item [\texttt{--save NAME [-d 'DESCRIPTION']}]: Searches the current directory for a \texttt{.config} file to be kept under the management of kw
\item [\texttt{--ls}]: List config files under kw management
\item [\texttt{--get NAME}]: Get a config file based named *NAME*
\item [\texttt{--rm}]: Remove config labeled with *NAME*
\end{description}

Example:
\begin{lstlisting}[language=bash]
cd KERNEL_PATH
kw g --save my_current_config
kw g --ls
kw g --get my_current_config
\end{lstlisting}

\section{Build}
\texttt{kw b} or \texttt{kw build}\\

\section{Deploy new Kernel/module}
\begin{description}
\item [\texttt{deploy,d}]: If you are in a kernel directory, this command will try to install the current kernel version in your target machine (remote, host, and VM).
\end{description}

Options:

\begin{description}
    \item [\texttt{--remote [REMOTE:PORT]}]: Specify the deploy to a remote machine.
    \item [\texttt{--local}]: Deploy in the host machine.
    \item [\texttt{--vm}]: Deploy in the QEMU vm.
    \item [\texttt{--reboot}]: Reboot machine after deploy.
    \item [\texttt{--modules}]: Only deploy modules.
\end{description}

\section{SSH}

\begin{description}
    \item [\texttt{ssh,s}]:
\end{description}

Options:

\begin{description}
    \item [\texttt{--script,-s [SCRIPT PATH]}]: Expects a bash script as a parameter to evaluate in a target machine.
    \item [\texttt{--command,-c=[COMMAND]}]: Expects a command to be executed in a target machine.
\end{description}

Example:
\begin{lstlisting}[language=bash]
kw s
kw s -c="dmesg -wH"
\end{lstlisting}

\section{\texttt{kworkflow.config} options}

\begin{description}
    \item [\texttt{ssh\_ip}]:
    Default ssh ip
    \item [\texttt{ssh\_port}]:
    Default ssh port
    \item [\texttt{arch}]:
    Specify the default architecture used by KW
    \item [\texttt{virtualizer}]:
    Defines the virtualization tool that should be used by kw. Current, we only support QEMU
    \item [\texttt{mount\_point}]:
    Defines the kw mount point, this directory is used by libguestfs during the mount/umount operation of a VM
    \item [\texttt{qemu\_hw\_options}]:
    Sets basic QEMU options
    \item [\texttt{qemu\_net\_options}]:
    Defines the network configuration
    \item [\texttt{qemu\_path\_image}]:
    Specify the VM image path
    \item [\texttt{alert}]:
    Default alert options (You should use vs, s, v or n. See README.md for details on this options)
    \item [\texttt{sound\_alert\_command}]:
    Command to run for sound completion alert (This command will be executed in the background)
    \item [\texttt{visual\_alert\_command}]:
    Command to run for visual completion alert (This command will be executed in the background)
    \item [\texttt{default\_deploy\_target}]:
    Sometimes it could be bothersome to pass the same parameter for kw deploy; here, you can set the default target. We define `vm` as the default, but you can also use `local` and `remote REMOTE:PORT`.
    \item [\texttt{reboot\_remote\_by\_default}]:
\end{description}

\end{multicols}

\end{document}

